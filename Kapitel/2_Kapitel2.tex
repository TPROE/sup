\section{Nummer 2}
\noindent Displayed equations are centered and set on a separate
line.
\begin{equation}
    x + y = z
\end{equation}
Please try to avoid rasterized images for line-art diagrams and
schemas. Whenever possible, use vector graphics instead (see
Fig.~\ref{fig1}).

\begin{figure}
    \includegraphics[width=\textwidth]{Bilder/enton.png}
    \caption{A figure caption is always placed below the illustration.
        Please note that short captions are centered, while long ones are
        justified by the macro package automatically. Vorher war hier eine eps DAtei} \label{fig1}
\end{figure}

\begin{theorem}
    This is a sample theorem. The run-in heading is set in bold, while
    the following text appears in italics. Definitions, lemmas,
    propositions, and corollaries are styled the same way.
\end{theorem}
%
% the environments 'definition', 'lemma', 'proposition', 'corollary',
% 'remark', and 'example' are defined in the LLNCS documentclass as well.
%
\begin{proof}
    Proofs, examples, and remarks have the initial word in italics,
    while the following text appears in normal font.
\end{proof}
For citations of references, we prefer the use of square brackets
and consecutive numbers. Citations using labels or the author/year
convention are also acceptable. The following bibliography provides
a sample reference list with entries for journal
\cite{ref_article1,ref_lncs1,ref_book1},
\cite{ref_article1,ref_book1,ref_proc1,ref_url1}.

\subsubsection{Acknowledgements} Please place your acknowledgments at
the end of the paper, preceded by an unnumbered run-in heading (i.e.
3rd-level heading).
\section{Fazit}
Die Fahrgastüberwachung im ÖPNV ist in vielerlei Hinsicht ein prekäres Thema, auch aus der Perspektive von Datenschutz und IT-Sicherheit.
Die einzige Kontrollmöglichkeit der betroffenen Personen über ihre Daten ist das Vermeiden der Verkehrsmittel, was für viele Menschen nicht umsetzbar ist.

Weiterhin gibt es oft nur wenig Transparenz vonseiten der Verkehrsbetriebe zu der Menge der gesammelten Daten sowie der Art der Verarbeitung. Über den Link des in Abbildung \ref{fig:hinweis}
dargestellten Schildes wird beispielsweise nur auf die Homepage des Verkehrsbetriebes verwiesen, ohne konkrete weiterführende Informationen bereitzustellen. Auch über die in Abschnitt \ref{abschnitt:technisch} beschriebene
Verwendung von Mikrofonen oder in Abschnitt \ref{abschnitt:massnahmen} vorgeschlagenen internen Prozesse zur Sicherstellung des Datenschutzes wird wenig aufgeklärt.

Diesen Mängeln gegenüber steht die in \glqq{}Abschnitt \ref{Risikoidentifizierung}: Adversary's Methods\grqq{} beschriebene große Angriffsfläche auf welche auf verschiedene Arten und mit unterschiedlichen Motivationen zugegriffen werden kann.
Eine Schmälerung des Risikos eines Angriffs selbst ist die Tatsache, dass die Videoüberwachung eines Verkehrsbetriebs selten ein lohnendes Ziel darstellt, verglichen mit anderen Unternehmen.


Die effektivsten Maßnahmen zur Sicherung der Fahrgastüberwachung umfassen also das Einrichten einer sicheren, verschlüsselten Verbindung und das Propagieren
einer gesunden Unternehmenskultur. Zu Letzterem gehört insbesondere eine Schulung der Mitarbeiter, bei welcher ein besonderes Augenmerk auf den verantwortungsbewussten Umgang
mit den aufgezeichneten Daten liegt.
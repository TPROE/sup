\section{Fazit}
Die Fahrgastüberwachung im ÖPNV ist aus vielerlei Hinsicht ein prekäres Thema. Die einzige Kontrollmöglichkeit über das Vermeiden einer Aufzeichnung ist die Vermeidung der Verkehrsmittel, was für viele Menschen
nicht umsetzbar ist.

Weiterhin gibt es oft nur wenig Transparenz von Seiten der Verkehrsbetriebe zu der Menge der gesammelten Daten sowie der Art der Verarbeitung. Über den Link des in Abbildung \ref{fig:hinweis}
dargestellten Schildes wird nur auf die Homepage des Verkehrsbetriebes verwiesen, ohne konkrete Informationen zu liefern. Auch über die in Abschnitt \ref{abschnitt:technisch} beschriebene
Verwendung von Mikrofonen oder in Abschnitt \ref{abschnitt:massnahmen} vorgeschlagenen internen Prozesse zur Sicherstellung des Datenschutzes wird wenig aufgeklärt.

Dem gegenüber steht die große Angriffsfläche, auf welche auf verschiedene Arten und mit unterschiedlicher Motivationen zugegriffen werden kann.
\section{Übersicht}
%In diesem Kapitel wird das Umfeld der Fahrgastüberwachung im ÖPNV genauer dargestellt.
\subsection{Abgrenzung des Begriffes} %weg?
Öffentlicher Personennahverkehr bezeichnet den \glqq{}räumlichen Bereich zur Beförderung von Personen im Berufs-,
Ausbildungs-, Einkaufs- und sonstigen alltäglichen Verkehr mit Fahrzeugen des Straßen-, Schienen- und Schiffsverkehrs (Fähren) im Linienverkehr.
\grqq{} \cite{Dr.FriedrichvonStackelbergDr.RobertMalina.2018}

Die Fahrgastüberwachung beschreibt die Maßnahmen, welche zur Aufzeichnung aller dem entsprechenden Verkehrsbetrieb zugehörigen Bereiche, in welchen sich Personen aufhalten, dienen.
Die Einsatzgebiete umfassen also die Haltestellen sowie, (Bus-)Bahnhöfe die Verkehrsmittel selbst.


\subsection{Rechtliche Rahmenbedingungen}
\label{abschnitt:rechtlich}
Die rechtliche Grundlage für die Fahrgastüberwachung ist durch § 4 Bundesdatenschutzgesetz Abs. 1 S. 2 \glqq{} Videoüberwachung öffentlich zugänglicher Räume\grqq{}
gegeben. Genauer heißt es dort:\\
\glqq{}Bei der Videoüberwachung von
\begin{enumerate}
    \item öffentlich zugänglichen großflächigen Anlagen, wie insbesondere Sport-,\\ Versammlungs- und Vergnügungsstätten, Einkaufszentren oder Parkplätzen, oder
    \item Fahrzeugen und öffentlich zugänglichen großflächigen Einrichtungen des öffentlichen Schienen-, Schiffs- und Busverkehrs
\end{enumerate}
gilt der Schutz von Leben, Gesundheit oder Freiheit von dort aufhältigen Personen als ein besonders wichtiges Interesse.\grqq{} \cite{Bundestag.2018}

Ebenfalls die Fahrgastüberwachung betreffend ist die europäische Datenschutzgrundverordnung, welche die Berechtigung zur Überwachung, Informationspflicht und das Recht auf Löschung festlegt:
\begin{table}
    \caption{Für die Fahrgastüberwachung im ÖPNV relevante Artikel der DSGVO \cite{EuropaischeUnion.}}\label{tab:dsgvo}
    \begin{tabular}{|l|l|l|}
        \hline
        Artikel                         & Bedeutung                                           \\
        \hline
        Art.6 DSGVO Abs.1. S.1 Nr. d--f & Rechtmäßigkeit der Verarbeitung besteht durch       \\
                                        & den Schutz lebenswichtiger Interessen (d), die im   \\
                                        & öffentlichen Interesse liegende Aufgabe (e) und das \\
                                        & berechtigte Interesse gemäß § 4 BDSG  Abs. 1 S. 2   \\ [1ex]
        Art. 13 DSGVO                   & Informationspflicht bei Erhebung von personen-      \\
                                        & bezogenen Daten bei der betroffenen Person          \\ [1ex]
        Art. 14 DSGVO                   & Informationspflicht bei Erhebung von personen-      \\
                                        & bezogenen Daten, die nicht bei der betroffenen      \\
                                        & Person erhoben wurden                               \\ [1ex]
        Art. 17 DSGVO                   & Recht auf Löschung                                  \\
        \hline
    \end{tabular}
\end{table}


\subsection{Technischer Hintergrund}
\label{abschnitt:technisch}
In Abbildung~\ref{fig:architektur} wird der Aufbau eines Systems zur Fahrgastüberwachung im ÖPNV dargestellt. Hierbei wird unterschieden zwischen den lokalen Komponenten, die an den Fahrzeugen und Haltestellen
verbaut sind, und der zentralen Leitstelle. Über eine Netzwerkverbindung ist jede lokale Einheit mit der Zentrale verbunden.

Ein lokales System hat mehrere Kameras verbaut. Falls ältere, analoge Aufnahmegeräte verbaut sind, muss deren Bild vor Anbindung an das Netzwerk digitalisiert werden.
Optional in den lokalen Systemen sind Anzeigen für die Kamerabilder sowie Mikrofone, welche in Kameras integriert sein können \cite{Reuter.12.02.2019}.
% z. B. den Fahrer des Busses. Da in Kameras teilweise Mikrofone integriert sind, können diese nicht ausgeschlossen werden.

In der zentralen Verwaltung stehen die Server, welche die Videoaufnahmen verwalten, sowie das \glqq{}Langzeit-Archiv\grqq{}. In diesem werden die Aufnahmen bis zur Löschung aufbewahrt.
Über den Zeitraum bis zur Löschung der Daten besteht bisher Uneinigkeit, gemäß § 27 Bundespolizeigesetz (BPolG) ist eine Speicherung von bis zu 30 Tagen zulässig. In  §6b Abs. 5 BDSG
steht dem die unweigerliche Löschung der Daten gegenüber. In der \glqq{}Orientierungshilfe zur Videoüberwachung\grqq{} der Datenschutzbeauftragten von Niedersachsen wird final eine maximale
Speicherdauer von 48 Stunden angegeben\cite{DanielaWindelband.20.April2016}. Neben der Darstellung der Videos sind auch technische Möglichkeiten zur Auswertung der Videos geschaffen.
Auswertung bezeichnet nicht automatisch die automatische Analyse der Daten, sondern beschreibt vielmehr das Betrachten und Beurteilen dieser durch einen Mitarbeiter.
In der zentralen Leitstelle, von welcher aus der Fahrbetrieb geregelt wird, ist häufig auch die Anzeige der Überwachungskameras.
\begin{figure}[ht]
    \begin{center}
        \includegraphics[width= 1\textwidth]{Bilder/architektur.png}
        \caption{Aufbau eines typischen Systems nach \cite{LandesbeauftragtefurdenDatenschutzBadenWurttemberg.2015}}
        \label{fig:architektur}
    \end{center}
\end{figure}

Die genaue Implementierung eines Systems kann abweichen, wichtig zur Betrachtung der Risiken sind aber die Aspekte der
Aufzeichnung an sich, die Datenübertragung über ein Netzwerk, die Bündelung in einer zentralen Leitstelle, temporäre Speicherung und die Möglichkeit zum mobilen Zugang.
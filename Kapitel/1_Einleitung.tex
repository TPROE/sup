\section{Einleitung}
Seit der Einführung der Videoüberwachung in den Bussen und Bahnen des Öffentlichen Personennahverkehrs (ÖPNV) sind diese Kameras ein Diskussionsthema in
der deutschen Gesellschaft. Neben dem Argument des kommenden Überwachungsstaates gegenüber der Forderung nach mehr Sicherheit im öffentlichen Raum gibt es auch noch die
Position der Verkehrsbetriebe, die Vandalismus in ihren Verkehrsmitteln vorbeugen möchten. Allerdings ist der Status quo eine allgemeine Überwachung aller Fahrgäste im ÖPNV,
wodurch sich für diese Personen Datenschutzrisiken ergeben.

Diese Hausarbeit im Fach \glqq{}Security und Privacy\grqq{} soll die Fahrgastüberwachung aus der Perspektive von Datenschutz und IT-Sicherheit darstellen und zugehörige Risiken mit entsprechenden
Maßnahmen erläutern. Das Ziel ist auch die Verwendung von in der Vorlesung besprochenen Methoden und Inhalten zur Identifikation der Risiken, Beschreiben der Szenarien und dem Ableiten der Maßnahmen.


Zur Risikoidentifizierung in der IT-Sicherheit werden \glqq{}Security-Cards\grqq{} verwendet und das Szenario wird durch ein \glqq{}Risiko-Register\grqq{} verdeutlicht. Zugehörige Maßnahmen werden aus
den Inhalten der Vorlesung ausgewählt. Für den Datenschutz werden die Risiken mit Hilfe der \glqq{}Seven Types of Privacy\grqq{} identifiziert, welche auch in der Beschreibung des Szenarios
Anwendung finden. Mögliche Maßnahmen werden durch die \glqq{}Privacy Design Strategies\grqq{} beschrieben. Die Funktionsweise der Methoden wird jeweils in dem entsprechenden Abschnitt dargelegt.


Es wird zuerst der Begriff der Fahrgastüberwachung selbst erklärt und anschließend die rechtlichen und technischen Rahmenbedingungen aufgezeigt. Im Kapitel \glqq{}Security\grqq{} werden für die
IT-Sicherheit die Risiken identifiziert, ein konkretes Szenario beschrieben und dazu passende Maßnahmen abgeleitet. Im Kapitel \glqq{}Privacy\grqq{} wird Selbiges für den Datenschutz wiederholt.
Abschließend wird ein Fazit zu den Möglichkeiten und Risiken in der Fahrgastüberwachung gezogen.

Aus Gründen der besseren Lesbarkeit wird auf die gleichzeitige Verwendung der Sprachformen männlich,
weiblich und divers (m/w/d) verzichtet. Sämtliche Personenbezeichnungen gelten gleichermaßen für alle Geschlechter.
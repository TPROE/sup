\section{Einleitung}
Seit der Einführung der Videoüberwachung in den Bussen und Bahnen des Öffentlichen Personennahverkehrs (ÖPNV) sind diese Kameras ein Diskussionsthema in
der deutschen Gesellschaft. Mit der Aufzeichnung von Personen einhergehend sind Risiken für den Datenschutz dieser Bilder,aber auch der Sicherheit des Systems selbst.


Diese Hausarbeit im Fach \glqq{}Security und Privacy\grqq{} soll das Umfeld der Fahrgastüberwachung darstellen sowie Risiken für Datenschutz und IT-Sicherheit mit zugehörigen Maßnahmen erläutern.
Das Ziel ist auch die Verwendung der in der Vorlesung besprochenen Methoden und Inhalte zur Identifikation der Risiken, Beschreiben der Szenarien und Ableiten der Maßnahmen.


Zur Risikoidentifizierung in der IT-Sicherheit werden \glqq{}Security-Cards\grqq{} verwendet und das Szenario wird durch ein \glqq{}Risiko-Register\grqq{} verdeutlicht. Maßnahmen werden aus
den Inhalten der Vorlesung ausgewählt. Für den Datenschutz werden die Risiken mittels der \glqq{}Seven Types of Privacy\grqq{} identifiziert, welche auch in der Beschreibung des Szenarios
Anwendung finden. Mögliche Maßnahmen werden mittels der \glqq{}Privacy Design Strategies\grqq{} beschrieben.


Es wird zuerst der Begriff der Fahrgastüberwachung selbst erklärt, und anschließend die rechtlichen und technischen Rahmenbedingungen aufgezeigt. Im Kapitel \glqq{}Security\grqq{} werden für die
IT-Sicherheit die Risiken identifiziert, ein konkretes Szenario beschrieben und dazu passende Maßnahmen abgeleitet. Im Kapitel \glqq{}Privacy\grqq{} wird selbiges für den Datenschutz wiederholt.
Abschließend wird ein Fazit über die Möglichkeiten und Risiken in der Fahrgastüberwachung gezogen.

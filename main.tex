% !TEX root = ./main.tex                             %WICHTIG FÜR VSCODE!!!!!!!!!!!!
\documentclass[runningheads]{llncs}
\bibliographystyle{splncs04}

%
\usepackage[T1]{fontenc}
\usepackage{graphicx}
\usepackage{comment}
\usepackage{hyperref}
\usepackage{color}
\usepackage{ngerman}

%\renewcommand\UrlFont{\color{blue}\rmfamily}


\begin{document}


\title{Fahrgastüberwachung im Öffentlichen Personennahverkehr}
\author{Tim Pröpper}
\institute{ Ostfalia Hochschule Wolfenbüttel, Am Exer 2, Wolfenbüttel, Braunschweig
    \email{info@ostfalia.de}\\
    \url{https://www.ostfalia.de}}

{\def\addcontentsline#1#2#3{}\maketitle}            % typeset the header of the contribution


\begin{abstract}
    Die Ausarbeitung beschäftigt sich mit den Risiken für den Datenschutz und die Informationssicherheit im Umfeld des
    Öffentlichen Personennahverkehrs.

    Der Fokus liegt hierbei auf der Betrachtung des Systems bestehend aus Kameras und Mikrofonen
    sowie der dahinterstehenden Architektur zur Verwaltung und potenziellen Auswertung. Hierfür wird zuerst ein Über\-blick über die komplette
    Architektur sowie das rechtliche Umfeld gegeben und anschließend werden alle möglichen Risiken bezüglich Privacy und Security gegenübergestellt.

    Es wird auf zwei konkrete Szenarien eingegangen:

    1. Ein Security-Risiko, welchem unerlaubt auf die Netzwerkverbindung zugegriffen wird. Hieraus entstehen verschiedene Bedrohungen,
    wie die Veröffentlichung oder Manipulation der Daten dieser Netzwerkverbin\-dung, oder auch das einfache Stören der Verbindung.
    2. Ein Privacy-Risiko, in welchem auf Videoaufnahmen unerlaubterweise Personen eindeutig identifiziert und die Aufnahmen geteilt werden.

    In diesen Szenarien wird der entstehende Schaden sowie beteiligte Komponenten dargestellt und anschließend werden zugehörige Maßnahmen erörtert.


    136/ (150--250) words.

    \keywords{ÖPNV  \and Überwachung \and Datenschutz.}
\end{abstract}

\setcounter{tocdepth}{2}
\tableofcontents
\setcounter{page}{0}
\section{Einleitung}
Seit der Einführung der Videoüberwachung in den Bussen und Bahnen des Öffentlichen Personennahverkehrs (ÖPNV) sind diese Kameras ein Diskussionsthema in
der deutschen Gesellschaft. Neben dem Argument des kommenden Überwachungsstaates gegenüber der Forderung nach mehr Sicherheit im öffentlichen Raum gibt es auch noch die
Position der Verkehrsbetriebe, die Vandalismus in ihren Verkehrsmitteln vorbeugen möchten. Allerdings ist der Status quo eine allgemeine Überwachung aller Fahrgäste im ÖPNV,
wodurch sich für diese Personen Datenschutzrisiken ergeben.

Diese Hausarbeit im Fach \glqq{}Security und Privacy\grqq{} soll die Fahrgastüberwachung aus der Perspektive von Datenschutz und IT-Sicherheit darstellen und zugehörige Risiken mit entsprechenden
Maßnahmen erläutern. Das Ziel ist auch die Verwendung von in der Vorlesung besprochenen Methoden und Inhalten zur Identifikation der Risiken, Beschreiben der Szenarien und dem Ableiten der Maßnahmen.


Zur Risikoidentifizierung in der IT-Sicherheit werden \glqq{}Security-Cards\grqq{} verwendet und das Szenario wird durch ein \glqq{}Risiko-Register\grqq{} verdeutlicht. Zugehörige Maßnahmen werden aus
den Inhalten der Vorlesung ausgewählt. Für den Datenschutz werden die Risiken mit Hilfe der \glqq{}Seven Types of Privacy\grqq{} identifiziert, welche auch in der Beschreibung des Szenarios
Anwendung finden. Mögliche Maßnahmen werden durch die \glqq{}Privacy Design Strategies\grqq{} beschrieben. Die Funktionsweise der Methoden wird jeweils in dem entsprechenden Abschnitt dargelegt.


Es wird zuerst der Begriff der Fahrgastüberwachung selbst erklärt und anschließend die rechtlichen und technischen Rahmenbedingungen aufgezeigt. Im Kapitel \glqq{}Security\grqq{} werden für die
IT-Sicherheit die Risiken identifiziert, ein konkretes Szenario beschrieben und dazu passende Maßnahmen abgeleitet. Im Kapitel \glqq{}Privacy\grqq{} wird Selbiges für den Datenschutz wiederholt.
Abschließend wird ein Fazit zu den Möglichkeiten und Risiken in der Fahrgastüberwachung gezogen.

Aus Gründen der besseren Lesbarkeit wird auf die gleichzeitige Verwendung der Sprachformen männlich,
weiblich und divers (m/w/d) verzichtet. Sämtliche Personenbezeichnungen gelten gleichermaßen für alle Geschlechter.
\section{Übersicht}
\subsection{Einsatzgebiet}
Öffentlicher Personennahverkehr bezeichnet den \glqq{} räumlichen Bereich zur Beförderung von Personen im Berufs-,
Ausbildungs-, Einkaufs- und sonstigen alltäglichen Verkehr mit Fahrzeugen des Straßen-, Schienen- und Schiffsverkehrs (Fähren) im Linienverkehr.
\grqq{} \cite{Dr.FriedrichvonStackelbergDr.RobertMalina.2018}
Die Einsatzgebiete von Fahrgastüberwachung betreffen alle dem entsprechenden Verkehrsbetrieb zugehörigen Gebiete, in welchen sich Personen aufhalten.
Dies umfasst also die Haltestellen sowie die Verkehrsmittel selbst.
\subsection{Rechtliche Rahmenbedingungen}
Die rechtliche Grundlage für die Fahrgastüberwachung ist durch §4 Bundesdatenschutzgesetz Abs. 1 S. 2 \glqq{} Videoüberwachung öffentlich zugänglicher Räume\grqq{}
gegeben. Genauer heißt es dort:\\
\glqq{}Bei der Videoüberwachung von
\begin{enumerate}
    \item öffentlich zugänglichen großflächigen Anlagen, wie insbesondere Sport-,\\ Versammlungs- und Vergnügungsstätten, Einkaufszentren oder Parkplätzen, oder
    \item Fahrzeugen und öffentlich zugänglichen großflächigen Einrichtungen des öffentlichen Schienen-, Schiffs- und Busverkehrs
\end{enumerate}
gilt der Schutz von Leben, Gesundheit oder Freiheit von dort aufhältigen Personen als ein besonders wichtiges Interesse.\grqq{} \cite{Bundestag.2018}\\
Ebenfalls die Fahrgastüberwachung betreffend ist die europäische Datenschutzgrundverordnung bezüglich Artikel 6 \glqq{}Rechtmäßigkeit der Verarbeitung\grqq{},
Artikel 13, 14\glqq{}Informationspflicht\grqq{} und Artikel 17 \glqq{}Recht auf Löschung\grqq{} \cite{EuropaischeUnion.}

\subsection{Architektur}
In Abbildung~\ref{fig:architektur} wird der Systemaufbau dargestellt. Hierbei wird unterschieden zwischen dem lokalen Aufbau, der an den Fahrzeugen und Haltestellen
anzufinden ist. Über ein Netzwerk ist jedes lokale System mit der zentralen Leitstelle verbunden.

Ein lokales System hat in der Regel mehrere Kameras (analog oder digital) verbaut. Das Bild analoger Kameras muss vor Anbindung an das Netzwerk digitalisiert werden.
Optional in den lokalen System sind eine Anzeige für z.B. deb Fahrer des Busses, sowie Mikrofone, welche in den Kameras integriert sein können.

In der zentralen Verwaltung stehen die Server, welche die Videoaufnahmen verwalten sowie das \glqq{}Langzeit-Archiv\grqq{}. In diesem werden die Aufnahmen bis zur Löschung aufbewahrt.
Über den Dauer bis zur Löschung der Daten heißt bisher Uneinigkeit, gemäß § 27 Bundespolizeigesetz (BPolG) ist eine Speicherung von bis zu 30 Tagen zulässig. In  §6b Abs. 5 BDSG
steht dem die unweigerliche Löschung der Daten gegenüber. In der \glqq{}Orientierungshilfe zur Videoüberwachung\grqq{} der Datenschutzbeauftragten von Niedersachsen wird eine maximale
Speicherdauer von 48 Stunden angegeben\cite{DanielaWindelband.20.April2016}. Neben der Darstellung der Videos sind auch technische Möglichkeiten zur Auswertung der Videos geschaffen.
Auswertung umfasst nicht automatisch die automatische Analyse der Daten, sondern beschreibt vielmehr das Betrachten und Beurteilen dieser durch einen Mitarbeiter.

\begin{figure}[ht]
    \begin{center}
        \includegraphics[width= 1\textwidth]{Bilder/architektur.png}
        \caption{Aufbau eines typischen Systems \cite{LandesbeauftragtefurdenDatenschutzBadenWurttemberg.2015}}
        \label{fig:architektur}
    \end{center}
\end{figure}
Die genaue Implementierung eines Systems kann abweichen, wichtig sind aber die Aspekte der Netzwerkübertragung und zentralen Auswertungsstelle für mehrere Kamerabilder.

\section{Security}
IT-Sicherheit

\subsection{Risikoidentifizierung}
\subsection{Konkretes Szenario}
\subsection{Maßnahmen}
\section{Privacy}
Privacy befasst sich mit dem Datenschutz der vom System betroffenen Personen. Im Falle der Fahrgastüberwachung im ÖPNV sind dies die Fahrgäste, Mitarbeiter der Verkehrsbetriebe sind hiervon ausgenommen.
\subsection{Risikoidentifizierung}
\label{abschnitt:7types}
Zur Identifikation der Risiken wird das Modell der \glqq{}Seven Types of Privacy\grqq{} verwendet, welche erstmals in dem Projekt \glqq{}Prescient\grqq{} beschrieben wurden \cite{Gutwirth.23.03.2011}.
Das Betrachten der verschiedenen Kategorien hilft bei der strukturierten Erfassung und Identifizierung von Risiken, die bei einer Verletzung des Datenschutzes entstehen können:
\begin{enumerate}
      \item {\bfseries Privacy of the Person} beschreibt die messbaren Körperwerte und -funk\-tionen wie Blutdruck, Körpertemperatur oder medizinische Informationen. Diese können bei der Fahrgastüberwachung
            im ÖPNV weder durch Kameras noch Mikrofone erfasst werden.
      \item {\bfseries Privacy of Behavior and Action}  bezeichnet die Informationen, die das persönliche Verhalten der Person beschreiben. Dies umfasst Gewohnheiten, Freizeitaktivitäten,
            Präferenzen aber auch die politische Einstellung. Durch das Nutzen des ÖPNV werden alle Gewohnheiten wie Arbeitszeiten und eventuelle Freizeitaktivitäten auf Video aufgezeichnet. Durch Mikrofone können
            ausgesprochene Meinungen dokumentiert werden. Dies hängt zusammen mit dem nächsten Stichpunkt:
      \item {\bfseries Privacy of Communication} betrifft die Kommunikation in allen Medien, wie direkte Gespräche, Telefonate oder schriftliche Kommunikation über das Internet. Gespräche können, falls Mikrofone installiert werden,
            aufgezeichnet werden.
      \item {\bfseries Privacy of Data and Image}  ist zutreffend bei der Videoüberwachung und Fotos, Videos im allgemeinen. Alle Kamerabilder der Fahrgastüberwachung fallen unter diesen Aspekt.
      \item {\bfseries Privacy of Thoughts and Feelings} lässt sich aus anderen Stichpunkten ableiten, da Gefühle und Gedanken nicht direkt gemessen werden können. Durch Körperhaltung, aufgezeichnete Gespräche,
            Gewohnheiten und gewöhnliche Aufenthaltsorte können aber Rückschlüsse gezogen werden.
      \item {\bfseries Privacy of Location and Space} verbietet das unerlaubte Aufzeichnen des Standortes einer Person. Im Falle der Fahrgastüberwachung ist durch die Kamerabilder, zugeordnet zugeordnet
            einer bestimmten Linie und versehen mit einem Zeitstempel, eine eindeutige Zuordnung der gefilmten Person zu einem Ort möglich.
      \item {\bfseries Privacy of Association} stellt Verbindungen zwischen einzelnen Personen und Gruppierungen her. Die \glqq{}Privacy of Association\grqq{} kann aus dem Inhalt der Kommunikation abgeleitet werden,
            aber auch aus dem gemeinsamen Aufenthaltsort von Personen. In der Videoüberwachung im ÖPNV könen die Kamerabilder Rückschlüsse auf Kommunikationspartner zulassen.
\end{enumerate}


\subsection{Konkretes Szenario}
\label{abschnitt:konkret}
Ein konkretes Szenario, bei dem der Datenschutz verletzt wird, ist das unerlaubte Teilen von Videoaufnahmen einer bestimmten Person. Dies kann auf verschiedene Weisen geschehen: Ein Miterarbeiter der in
Kapitel \ref{abschnitt:technisch} beschriebenen Zentrale entdeckt durch Zufall einen Bekannten auf dem Weg zu einer Veranstaltung, oder wird durch dritte dazu gezwungen nach einer bestimmten Person zu suchen.

Um das Risiko strukturiert analysieren zu können, wird eine Methode aus der Vorlesung verwendet. In dieser werden die betroffenen Personen und \glqq{}Seven Types of Privacy\grqq{} sowie das Szenario selbst beschrieben
und anschließend die beteiligten Personen sowie mögliche Folgen und Gründe dargestellt.
\begin{figure}[ht]
      \begin{center}
            \includegraphics[width= 0.76\textwidth]{Bilder/privacy.png}
            \caption{Anwendung der Methode auf das konkrete Szenario}
            \label{fig:Privacy}
      \end{center}
\end{figure}
\newline
Direkt betroffen ist in dem Szenario die Person auf den Kamerabildern eindeutig identifiziert wurde. Alle gemäß Abschnitt \ref{abschnitt:7types} beschriebenen Aspekte können in diesem Szenario angewandt werden.
Stakeholder sind in dem Szenario das soziale Umfeld der Person, aber auch die Person die die Identifikation durchgeführt hat sowie deren Verantwortlichen. Der konkrete Schaden der entsteht, ist ein Rückschluss
der auf das Verhalten der Person zugelassen wird, aber auch rechtliche oder gesellschatliche Konsequenzen können nicht ausgeschlossen werden. Zu einem solchen Verstoß kann es durch mangelnde Kontrolle oder falsche
Einflusspersonen/ Vorgesetzte in dem Unternehmen kommen.

\subsection{Maßnahmen}
\label{abschnitt:massnahmen}
Maßnahmen können mittels der \glqq{}Privacy Design Strategies\grqq{} (PDS) abgeleitet werden \cite{Hoepman.2022}. Diese beschreiben Herangehensweisen, um Datenschutz zu gewährleisten.
Aus den möglichen Strategien ist die erste Maßnahme, den Fahrgast rechtzeitig über die Aufzeichnung zu informieren. Eine übliche Möglichkeit ist in Abbildung \ref{fig:hinweis} dargestellt, das Schild hängt im Eingangsbereich der Bahn
und auf auf der Internetseite findet sich ein Kontaktformular. Damit zusammenhängend sind die Strategien \glqq{}Enforce\grqq{} und \glqq{}Demonstrate\grqq{}. Wie in Kapitel \ref{abschnitt:konkret} beschrieben,
ist eine solche Art von Vorfall oft das Resultat durch ungenügenden Umgang mit Daten in der Unternehmenskultur. Nach einem Vorfall ist es entsprechend wichtig, ausgehend vom Management das Thema Datenschutz
stärker zu thematisieren. Dieses Vorgehen wird durch die PDS \glqq{}Enforce\grqq{} beschrieben. Damit einhergehend ist die Strategie \glqq{}Demonstrate\grqq{}, in welcher die getroffenen
Maßnahmen dokumentiert und nach außen präsentiert werden. Eine Kontrolle über die Daten gemäßs PDS \glqq{}Control\grqq{} kann in der Videoüberwachung allerdings nicht ermöglicht werden.

Neben den prozessorientierten PDS gibt es noch datenorientierte PDS. Eine anwendbare ist die PDS \glqq{}Hide\grqq{}. Diese beschreibt die Einschränkung des Zugangs zu den Videoaufzeichnungen durch Zugangsbeschränkungen
aber auch durch Verschlüsselung von Daten. Weniger anwendbar dagegen ist die PDS \glqq{}Separate\grqq{}. Diese beschreibt das kontextabhängige Aufteilen von Daten, um kein Schließen von Korrelationen zuzulassen.
Allerdings ist das Zuordnen der Kamerabilder zu den Orten oder Fahrzeugen essentiell. Auch das Minimieren und Abstrahieren von Daten ist nur schwerlich anwendbar, da eine Videoaufzeichnung versehen mit einem Zeitstempel
schon eine geringe Form der Datenerhebung darstellt.
\begin{figure}[ht]
      \begin{center}
            \includegraphics[width= 0.5\textwidth]{Bilder/hinweis.png}
            \caption{Hinweis in einer Straßenbahn der Braunschweiger Verkehrs-GmbH.\\Eigene Fotografie, 19.01.2023}
            \label{fig:hinweis}
      \end{center}
\end{figure}

\section{Fazit}
Die Fahrgastüberwachung im ÖPNV ist aus vielerlei Hinsicht ein prekäres Thema. Die einzige Kontrollmöglichkeit über das Vermeiden einer Aufzeichnung ist die Vermeidung der Verkehrsmittel, was für viele Menschen
nicht umsetzbar ist.

Weiterhin gibt es oft nur wenig Transparenz von Seiten der Verkehrsbetriebe zu der Menge der gesammelten Daten sowie der Art der Verarbeitung. Über den Link des in Abbildung \ref{fig:hinweis}
dargestellten Schildes wird nur auf die Homepage des Verkehrsbetriebes verwiesen, ohne konkrete Informationen zu liefern. Auch über die in Abschnitt \ref{abschnitt:technisch} beschriebene
Verwendung von Mikrofonen oder in Abschnitt \ref{abschnitt:massnahmen} vorgeschlagenen internen Prozesse zur Sicherstellung des Datenschutzes wird wenig aufgeklärt.

Dem gegenüber steht die große Angriffsfläche, auf welche auf verschiedene Arten und mit unterschiedlicher Motivationen zugegriffen werden kann.

\bibliography{literatur}

\end{document}